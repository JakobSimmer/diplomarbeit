
Das Web-Interface kann in einem Browser über folgende URL abgerufen werden: \verb|http://localhost:3001|.
Es ist wichtig, dass \verb|localhost| verwendet wird und nicht die IP des Docker-Containers, da 
Probleme mit der CORS-Policy entstehen können. Alle Graphen sind untereinander angeordnet; 
um die individuellen Kosten anzupassen, kann auf den jeweiligen Graphen geklickt werden, 
und im Eingabefeld können die Kosten in Cent eingegeben werden. Um die Datenbank verwalten zu können, 
wurde PHPmyAdmin verwendet, welches auf folgender URL abgerufen werden kann: \verb|http://localhost:81|.
Die Anmeldedaten für die Datenbank lauten: 
\\
User: \verb|root|
\\
Password: \verb|RZp7z1FNp3atzHth|
\\
Datenbank: \verb|sensor|

\section{Mögliche Fehlerquellen}

Sollte das RaspberryPi neu gestartet werden, kann es vorkommen, dass sich die IP-Adresse der Datenbank
innerhalb des Docker Containers verändert hat. Im root-Verzeichnis des Projekts (\verb|~/Documents/diplomarbeit/|) befindet sich ein 
Skript namens \verb|init_db_ip.sh|. Dieses aktualisiert die IP-Adresse der Datenbank im Server. 
Es sollten folgende Schritte hintereinander ausgeführt werden, um die API (den Server) neu zu starten:
\\
\\
\verb|$> cd ~/Documents/diplomarbeit|
\\
\verb|$> ./init_db_ip.sh|
\\
\verb|$> sudo docker stop diplomarbeit_api|
\\
\verb|$> sudo docker remove diplomarbeit_api|
\\
\verb|$> cd web/|
\\
\verb|$> sudo docker compose up -d|
\\
\\
Falls die Datenbank noch nicht läuft, sollte das Docker Compose projekt angestartet werden:
\\
\verb|$> cd ~/Documents/diplomarbeit/web|
\\
\verb|$> sudo docker compose up -d|
\\
\\
Ob jeder Container erfolgreich läuft, kann mit folgendem Befehl überprüft werden:
\\
\verb|$> sudo docker ps -a|
\\
\\
Sollten keine Graphen angezeigt werden, kann über folgendem Befehl bei der API auf Fehlernachrichten überprüft werden:
\\
\verb|$> sudo docker logs diplomarbeit_api|

